\section{Discussion}\label{discussion}
Our results show that neural networks can be used to improve targeted alignment problem, which increases the efficiency of structural variation identification. Specifically, we have focused on the problem of genotyping tandem repeat variations where identification of informative reads using sensitive alignments takes up to half of the running time. Through extensive experiments, we show that neural networks can replace current graph based methods to tackle read selection problem. However, similar to targeted alignment problem, our approach is also limited to targets and cannot replace whole alignment process. Thus, while this approach could be used to improve alignment to specific regions, it cannot cover all types of variations through genome and each network is being trained specific to location of genome and possible type of variation.
\paragraph{Acknowledgements.} ack.
%\section{Acknowledgements}
